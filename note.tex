% $Id: main.tex 25184 2012-09-13 15:00:47Z mneedham $
% ===============================================================================
% Purpose: Template for LHCb documents
% Authors: Tomasz Skwarnicki, Roger Forty, Ulrik Egede
% Created on: 2010-09-24
% ===============================================================================
\documentclass[12pt,a4paper]{article}

% Variables that controls behaviour
%\usepackage{txfonts}
%%\usepackage{amssymb}
%\usepackage{slashbox}
\usepackage{mathtools}
%\usepackage{epstopdf}
\usepackage{booktabs}
\usepackage{epsfig}
\usepackage{multirow}
%%%%%
\usepackage{lineno}  % for line numbering during review
\usepackage{graphicx}  % to include figures (can also use other packages)
\usepackage{xspace} % To avoid problems with missing or double spaces after
                    % predefined symbold
\usepackage{color}
\usepackage{colortbl}
\usepackage{amsmath} % Adds a large collection of math symbols

\usepackage{amssymb}
\usepackage{amsfonts}
\usepackage{upgreek} % Adds in support for greek letters in roman typeset

% Get hyperlinks to captions and in references.
% These do not work with revtex. Use "hypertext" as class option instead.
\usepackage{hyperref}    % Hyperlinks in references
\usepackage[all]{hypcap} % Internal hyperlinks to floats.

% Make this the last package you include before the \begin{document}
\usepackage{cite}
\usepackage{mciteplus}



\usepackage{ifthen} % for conditional statements
\newboolean{pdflatex}
\setboolean{pdflatex}{true} % False for eps figures 

\newboolean{articletitles}
\setboolean{articletitles}{true} % False removes titles in references

\newboolean{uprightparticles}
\setboolean{uprightparticles}{false} %True for upright particle symbols
%
\newcommand{\upsmm}{\nobreak{\ensuremath{\varUpsilon\rightarrow \mu^+\mu^-}}}
\newcommand{\ups}{\ensuremath{\varUpsilon}}
\newcommand{\upsns}{\ensuremath{\varUpsilon(nS)}}
\newcommand{\ones}{\ensuremath{\varUpsilon(1S)}}
\newcommand{\twos}{\ensuremath{\varUpsilon(2S)}}
\newcommand{\threes}{\ensuremath{\varUpsilon(3S)}}
\newcommand{\ismm}{\ensuremath{\varUpsilon(nS)\rightarrow\mu^{+}\mu^{-}}}
\newcommand{\onesmm}{\ensuremath{\varUpsilon(1S)\rightarrow\mu^+\mu^-}}
\newcommand{\twosmm}{\ensuremath{\varUpsilon(2S)\rightarrow\mu^+\mu^-}}
\newcommand{\threesmm}{\ensuremath{\varUpsilon(3S)\rightarrow\mu^+\mu^-}}
\newcommand{\bmm}{\ensuremath{\mathcal{B}(\mu^+\mu^-)}}
\newcommand{\bmmis}{\ensuremath{\mathcal{B}(\varUpsilon(nS)\rightarrow\mu^+\mu^-)}}
\newcommand{\bmmones}{\ensuremath{\mathcal{B}(\varUpsilon(1S)\rightarrow\mu^+\mu^-)}}
\newcommand{\bmmtwos}{\ensuremath{\mathcal{B}(\varUpsilon(2S)\rightarrow\mu^+\mu^-)}}
\newcommand{\bmmthrees}{\ensuremath{\mathcal{B}(\varUpsilon(3S)\rightarrow\mu^+\mu^-)}}
\newcommand{\jpsi}{\ensuremath{$J/\psi$}}
\newcommand{\bit}{\begin{itemize}}
\newcommand{\bce}{\begin{center}}
\newcommand{\eit}{\end{itemize}}
\newcommand{\ece}{\end{center}}

\begin{document}


%%%%%%%%%%%%%%%%%%%%%%%%%
%%%%% Title     %%%%%%%%%
%%%%%%%%%%%%%%%%%%%%%%%%%
\renewcommand{\thefootnote}{\fnsymbol{footnote}}
\setcounter{footnote}{1}

% %%%%%%% CHOOSE --------
% $Id: title-LHCb-ANA.tex 26327 2012-10-09 14:49:13Z mneedham $
% ===============================================================================
% Purpose: LHCb-ANA Note title page template
% Author: 
% Created on: 2010-10-05
% ===============================================================================

%%%%%%%%%%%%%%%%%%%%%%%%%
%%%%%  TITLE PAGE  %%%%%%
%%%%%%%%%%%%%%%%%%%%%%%%%
\begin{titlepage}

% Header ---------------------------------------------------
\vspace*{-1.5cm}

\hspace*{-0.5cm}
\begin{tabular*}{\linewidth}{lc@{\extracolsep{\fill}}r}
\ifthenelse{\boolean{pdflatex}}% Logo format choice
{\vspace*{-2.7cm}\mbox{\!\!\!\includegraphics[width=.14\textwidth]{lhcb-logo.pdf}} & &}%
{\vspace*{-1.2cm}\mbox{\!\!\!\includegraphics[width=.12\textwidth]{lhcb-logo.eps}} & &}
 \\
 & & LHCb-ANA-20XX-YYY \\  % ID 
 & & \today \\ % Date - Can also hardwire e.g.: 23 March 2010
 & & version 0.0\\
\hline
\end{tabular*}

\vspace*{4.0cm}

% Title --------------------------------------------------
{\bf\boldmath\huge
\begin{center}
Observation of the decay $B_{u}^{+} \to J/\psi K^{+} K^{-} \pi^{+}$
\end{center}
}

\vspace*{2.0cm}
% Authors -------------------------------------------------
\begin{center}
A.~Baranov$^1$, V.~Belyaev$^1$, V.~Egorychev$^1$ \\
{\it\footnotesize
$^1$ITEP, Russia\\
}
\end{center}


\vspace{\fill}

% Abstract -----------------------------------------------
\begin{abstract}
  \noindent
State what you measure and the data sample you use.
\end{abstract}

\vspace*{2.0cm}
\vspace{\fill}

\end{titlepage}


\pagestyle{empty}  % no page number for the title 

%%%%%%%%%%%%%%%%%%%%%%%%%%%%%%%%
%%%%%  EOD OF TITLE PAGE  %%%%%%
%%%%%%%%%%%%%%%%%%%%%%%%%%%%%%%%

%  empty page follows the title page ----
\newpage
\setcounter{page}{2}
\mbox{~}
\begin{tabular}{|c|c|l|}
\hline
{\sc version} & {\sc date} & {\sc comments}\\
\hline
0.0 & 06.06.2012 & First version, about efficiencies\\
\hline
% 0.1 & 07.06.2012 & First version to the referees\\
% \hline
% 0.2 & 11.06.2012 & Added plots and tables\\
% \hline
\end{tabular}

\cleardoublepage

% \input{title-LHCb-CONF}
% \input{title-LHCb-PAPER}
% %%%%%%%%%%%%% ---------

\renewcommand{\thefootnote}{\arabic{footnote}}
\setcounter{footnote}{0}

%%%%%%%%%%%%%%%%%%%%%%%%%%%%%%%%
%%%%%  Table of Content   %%%%%%
%%%%%%%%%%%%%%%%%%%%%%%%%%%%%%%%
%%%% Uncomment next 2 lines if desired
\tableofcontents
\cleardoublepage


%%%%%%%%%%%%%%%%%%%%%%%%%
%%%%% Main text %%%%%%%%%
%%%%%%%%%%%%%%%%%%%%%%%%%

\pagestyle{plain} % restore page numbers for the main text
\setcounter{page}{1}
\pagenumbering{arabic}

% %%%%%%% CHOOSE --------
%% ----------------------------------
%% Line numbering on the left margin 
%% ----------------------------------
%% Uncomment during review phase. 
%% Comment it out before a final submission.
\linenumbers
%% --------------------------------
% %%%%%%%%%%%%% ---------


% You can include short sections directly in the main tex file.
% However, for larger papers it is desirable to split the text into
% several semiautonomous files, which can be revised independently.
% This is especially useful when developing a document in
% collaboration with several people, since then different parts can be
% edited independently.  This type of file organization is shown here.
% 

% \section{Introduction}
In this section state clearly what you are measuring, 
and how. Which channels you include etc.

% \section{Differences}
Here you can highlight the main differences with the 
previous versions, which can replace (or be replaced by) 
the table on the 
first page.


% \section{Dataset}
Please list clearly the data you are using, 
and all the MC samples, with possibly 
the number of events, the reconstruction version, 
the stripping version, the magnet polarities
used.

% \section{Trigger}
Always specify which trigger lines you 
select for your analysis, and if you know add a 
brief reminder of the cuts applied there.

% \section{Selection}
% This is a very crucial section. We would like to 
% see the selection clearly described, and possibly 
% a set of distributions of the variables 
% used in the selection, in data and MC.
% If you like, you can also add comparisons Mag Down vs. 
% Mag Up and 2011 vs. 2012 data.
% These checks are welcome but not required.



WG selection on K, $\pi$:

\vspace*{0.5cm}
\mbox{~}
\begin{tabular}{|c|c|}
\hline
Variable & Cut \\
\hline

$P_{T}$ & $> 200 MeV/c$ \\
CLONEDIST  & $> 5000$ \\
Ghost Prob. & $< 0.5$ \\
Track $\chi^{2}/ndof$ & $< 4$ \\
$\eta$ & $\in (2, 5)$ \\
P & $\in (3.2,150) GeV/c$\\
$Prob. NN$ & $> 0.1$\\
min $\,\chi^{2}_{IP}$ & $> 4$\\

\hline
\end{tabular}
\vspace*{0.5cm}

WG selection on B:

\vspace*{0.5cm}
\mbox{~}
\begin{tabular}{|c|c|}
\hline
Variable & Cut \\
\hline
$\chi^{2}_{vtx}$ & $< 12$ \\
$c\tau$ & $> 75 \mu m$ \\

\hline
\end{tabular}
\vspace*{0.5cm}

Custom selection:

\vspace*{0.5cm}
\mbox{~}
\begin{tabular}{|c|c|}
\hline
Variable & Cut \\
\hline
$\chi^{2}_{DTF} / ndof$ (*) & $\in (0, 5)$ \\
$c\tau(B^{+})$ (*) & $> 250 \mu m$ \\
$\chi^{2}_{vtx.}(B^{+})$ & $< 20$ \\
$p_{T}(\pi)$ & $> 0.3 GeV/c$ \\
$p_{T}(K)$ & $> 0.6 GeV/c$ \\
$M_{J/\psi}$ & $\in (3.020, 3.135) GeV/c^2$ \\
$Prob.NN\,of\,K$ & $> 0.3$ \\
$Prob.NN\,of\,\pi$ & $> 0.3$ \\
$min \chi^{2}_{IP}(K, \pi)$ & $> 12$ \\
\hline
\end{tabular}
\vspace*{0.5cm}


Trigger:

\vspace*{0.5cm}
\mbox{~}
\begin{tabular}{|c|c|c|}
\hline
L0 & HLT1 & HLT2 \\
\hline
DiMuon& DiMuon & DiMuon \\
Muon& SingleMuon & ExpressJPsi \\
& TrackMuon & SingleMuon \\
\hline
\end{tabular}
\vspace*{0.5cm}


\section{Efficiencies}
% It is nice to add here the method you use to 
% estimate the efficiency, and at least a plot of the 
% efficiency(ies) itself.

The overall efficiency is the product of the generator level efficiency ($\epsilon^{gen\&acc}$) which includes the geometrical acceptance of the detector and generator level cuts efficiency, the combined reconstruction and selection efficiency ($\epsilon^{rec\&sel}$) and the trigger efficiency ($\epsilon^{trig}$). For first approach we are using $\epsilon^{full} = \epsilon^{rec\&sel} \times \epsilon^{trig}$.


\subsection{Acceptance efficiency}
Obtained individual generator level efficiencies for signal and normalization channels is presented in the table below:


\vspace*{0.5cm}
\mbox{~}
\begin{tabular}{|l|l|c|c|}
\hline
Year & Magnet & $B_{u} \to J/\psi KK\pi$ & $B_{u} \to \psi' K$ \\
\hline
2011 & Up   & $0.15678 \pm 0.00041$ & $0.15311 \pm 0.00039$ \\
2011 & Down & $0.15713 \pm 0.00039$ & $0.15390 \pm 0.00039$ \\
2012 & Up   & $0.16175 \pm 0.00060$ & $0.1559  \pm 0.000393$ \\
2012 & Down & $0.1609  \pm 0.0013$  & $0.1562  \pm 0.000391$ \\
\hline
\end{tabular}
\vspace*{0.5cm}

The average of four ratios is $\epsilon_{gen\&acc}^{KK\pi} / \epsilon_{gen\&acc}^{\psi'K} = 1.028 \pm 0.003$ is chosen as the resulting ratio of geometrical acceptance efficiencies (uncertainty is statistical only). The maximal devistion from the average is taken as systematic uncertainty and equals $1\%$.

\subsection{Full efficiency}

For the signal channel we have $N_{MC-input}^{KK\pi} = 1538248$ events. Number of events was obtained from get-bookkeeping-info and by grepping Ganga's output. On whole all generated events we apply reconstruction, WG selection and custom selection criteria with trigger cuts and MC-true included. After applying all selection criteria we get distribution on Figure ~\ref{fig:mc-signal}, which is fitted by Double-Sided Crystal Ball function with all parameters set to be free. We get $N_{mc-sig} = 22394 \pm 149$ events. Now we can calculate 
$$\epsilon^{full}_{sig} =  \frac { N_{mc-sig} }{ N_{MC-input}^{KK\pi} } =  (1.46 \pm 0.01) \times 10^{-2}$$

Same procedure for normalization channel. Here we have $N_{MC-input}^{\psi'K} = 3673304$ events. Fit is on Figure ~\ref{fig:mc-norm}, from it we get $N_{mc-norm} = 32775 \pm 183$. And the efficiency is 
$$\epsilon^{full}_{norm} =  \frac { N_{mc-norm} }{ N_{MC-input}^{\psi'K} } =  (0.892 \pm 0.005) \times 10^{-2}$$


\pagebreak
\subsection{Braching ratio}

Now we can obtain braching ratio:

$$
\frac{\mathcal{B}(B_{u} \to J/\psi \, KK\pi)}{\mathcal{B}(B_{u} \to (\psi(2S) \to J/\psi \pi\pi) \, K)} = \
\frac{ N_{sig.} }{ N_{norm.} } \times \frac{\epsilon_{gen\&acc}^{KK\pi} }{ \epsilon_{gen\&acc}^{\psi'K}} \times \frac{\epsilon_{sig.}^{full}}{\epsilon_{norm.}^{full}} =
$$
$$
= \frac{224 \pm 26}{9153 \pm 68} \times (1.028 \pm 0.003) \times \frac{(1.46 \pm 0.01) \times 10^{-2}}{(0.892 \pm 0.005) \times 10^{-2}} = \
0.041 \pm 0.005 = \mathcal{R}
$$

With ratio $\mathcal{R}$ we can now obtain: 
$$
\mathcal{B}(B_{u} \to J/\psi \, KK\pi) = \mathcal{B}(B_{u} \to \psi(2S) K) \times \mathcal{B}(\psi(2S) \to J/\psi \pi\pi) \times \mathcal{R} \simeq
$$

$$
\simeq 6.27 \times 10^{-4} \times 0.34 \times \mathcal{R} = (8 \pm 1) \times 10^{-6}
$$


\begin{figure}[here]
\centering
\includegraphics[width = 0.8\textwidth]{img/MC-signal.png}
\caption{Signal MC distribution with all cuts applied}
\label{fig:mc-signal}
\end{figure}


\begin{figure}[here]
\centering
\includegraphics[width = 0.8\textwidth]{img/MC-norm.png}
\caption{Normalization channel MC distribution with all cuts applied}
\label{fig:mc-norm}
\end{figure}




% \input{src/Systematics.tex}

% \section{Results}
Here add your nice final results, plots, and table, 
and their discussion~\cite{jp}.


% \section{Acknowledgements}
Do not forget to thank who helped you! 


\section{Appendix}
Here put all the plots which might be too heavy for the 
main body. Any important discussion should however be included
in the main part of the note.
\\

\addcontentsline{toc}{section}{References}
\bibliographystyle{LHCb}
\bibliography{lhcb-ana-BandQ-example}
Using bibtex will make your life easier later!
\end{document}

